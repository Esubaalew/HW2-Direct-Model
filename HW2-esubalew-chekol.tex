\documentclass[11pt,a4paper]{article}

%----------------------------------------------------------------------------------------
%   PACKAGES AND OTHER DOCUMENT CONFIGURATIONS
%----------------------------------------------------------------------------------------

\usepackage[T1]{fontenc}
\usepackage{lmodern}
\usepackage{amsmath, amssymb, amsthm, amsfonts}
\usepackage{geometry}
\geometry{a4paper, margin=1in, top=25mm, bottom=25mm}

\usepackage{xcolor}
\usepackage{tcolorbox}
\tcbuselibrary{skins, breakable}
\usepackage{enumitem}
\usepackage{microtype}
\usepackage{fancyhdr}
\usepackage{background}
\usepackage{hyperref}
\usepackage{fontawesome}

%----------------------------------------------------------------------------------------
%   COLOR DEFINITIONS
%----------------------------------------------------------------------------------------

\definecolor{UMBlue}{RGB}{0,39,76}
\definecolor{UMMaize}{RGB}{255,203,5}
\definecolor{LightGray}{RGB}{245,245,245}
\definecolor{BoxBlue}{RGB}{220,235,245}
\definecolor{AccentYellow}{RGB}{255,240,200}

%----------------------------------------------------------------------------------------
%   PAGE STYLING
%----------------------------------------------------------------------------------------

\backgroundsetup{
contents={},
color=LightGray,
opacity=0.1,
scale=1,
angle=0
}

\pagestyle{fancy}
\fancyhf{}
\renewcommand{\headrulewidth}{0pt}
\fancyhead[L]{\textcolor{UMBlue}{\large\textbf{Graphical Models Worksheet}}}
\fancyhead[R]{\textcolor{UMBlue}{\large\textbf{Esubalew C. Muluye}}}
\fancyfoot[C]{\textcolor{UMBlue}{\thepage}}

%----------------------------------------------------------------------------------------
%   CUSTOM ENVIRONMENTS
%----------------------------------------------------------------------------------------

\newtcolorbox{exercisebox}[1]{
    breakable,
    enhanced,
    colback=BoxBlue,
    colframe=UMBlue,
    title=#1,
    fonttitle=\bfseries\large\color{UMBlue},
    attach title to upper,
    boxrule=1pt,
    arc=4pt,
    before skip=12pt,
    after skip=12pt,
    overlay unbroken and first={
        \node[anchor=north east] at (frame.north east) 
            {\color{UMBlue!60!white}\scalebox{1.5}{\faPencil}};
    }
}

\newenvironment{solution}
    {\par\medskip\noindent\textbf{\color{UMBlue}Solution}\quad\color{DarkGray}}
    {\par\medskip}

%----------------------------------------------------------------------------------------
%   TITLE SECTION
%----------------------------------------------------------------------------------------

\title{\sffamily\bfseries\Huge Worksheet 1: Graphical Models \\ 
\Large Solutions \vspace{-0.2cm}}
\author{\sffamily\Large Esubalew Chekol Muluye \\ \small ID: GSR/6451/17}
\date{\sffamily\today}

%----------------------------------------------------------------------------------------
%   DOCUMENT CONTENT
%----------------------------------------------------------------------------------------

\begin{document}

\maketitle
\thispagestyle{empty}

\vspace{-1cm}
\begin{center}
\rule{0.8\textwidth}{0.5pt}
\end{center}

\clearpage

\begin{exercisebox}{Exercise 3: Factorization \& Independence}
Let $X, Y, Z$ be three disjoint subsets of variables such that $x = X \cup Y \cup Z$. Prove that $P \models (X \indep Y | Z)$ if and only if we can write $P$ in the form: $P(x) = P(X,Y,Z) = \phi_1(X,Z) \cdot \phi_2(Y,Z)$.

\begin{solution}
\textbf{Forward Direction ($\Rightarrow$):}
\begin{align*}
P(X,Y,Z) &= \condprob{X,Y}{Z}P(Z) \\
&= \condprob{X}{Z}\condprob{Y}{Z}P(Z) \quad \text{(by independence)} \\
&= \underbrace{\condprob{X}{Z}}_{\phi_1(X,Z)} \cdot \underbrace{\condprob{Y}{Z}P(Z)}_{\phi_2(Y,Z)}
\end{align*}

\textbf{Reverse Direction ($\Leftarrow$):}
\begin{align*}
\condprob{X}{Y,Z} &= \frac{\phi_1(X,Z)\phi_2(Y,Z)}{\sum_X \phi_1(X,Z)\phi_2(Y,Z)} \\
&= \frac{\phi_1(X,Z)}{\sum_X \phi_1(X,Z)} = \condprob{X}{Z}
\end{align*}
Thus $X \indep Y | Z$ holds.
\end{solution}
\end{exercisebox}

\begin{exercisebox}{Exercise 4: Reasoning by Cases}
Prove: $\condprob{X}{Y} = \sum_z \condprob{X, Z=z}{Y}$ using chain rule and conditional probability properties.

\begin{solution}
Using law of total probability:
\begin{align*}
\condprob{X}{Y} &= \sum_z \condprob{X,Z=z}{Y} \\
&= \sum_z \frac{\prob{X,Y,Z=z}}{\prob{Y}} \\
&= \frac{1}{\prob{Y}} \sum_z \prob{X,Y,Z=z} \\
&= \frac{\prob{X,Y}}{\prob{Y}} = \condprob{X}{Y}
\end{align*}
\end{solution}
\end{exercisebox}

\begin{exercisebox}{Exercise 5: Conditional Independence Properties}
Let $W, X, Y, Z$ be sets of random variables.

\subsection*{(a) Weak Union \& Contraction}
Prove: $(X \indep Y,W | Z) \Rightarrow (X \indep Y | Z,W)$ and contraction property.

\subsection*{(b) Intersection Property}
Prove for positive distributions: $(X \indep Y | Z,W) \land (X \indep W | Z,Y) \Rightarrow (X \indep Y,W | Z)$

\subsection*{(c) Counterexample}
Provide non-positive distribution counterexample for intersection property.

\begin{solution}
\textbf{(a) Weak Union:}
\begin{align*}
\condprob{X}{Y,Z,W} &= \condprob{X}{Z} \quad \text{(given)} \\
&= \condprob{X}{Z,W} \quad \text{(by decomposition)}
\end{align*}

\textbf{Contraction:} Combine two independence statements through:
\begin{align*}
\condprob{X}{Y,W,Z} &= \condprob{X}{Y,Z} \quad \text{(given)} \\
&= \condprob{X}{Z} \quad \text{(given)}
\end{align*}

\textbf{(b) Intersection:} Use positivity to show:
\begin{align*}
\condprob{X}{Z,W} &= \condprob{X}{Z,Y} \\
\Rightarrow \condprob{X}{Z} &= \condprob{X}{Z,Y} \Rightarrow (X \indep Y | Z)
\end{align*}

\textbf{(c) Counterexample:} Construct distribution where:
\begin{itemize}
\item $X=Y=W$ with $P(Z=0)=1$
\item $(X \indep Y | W,Z)$ and $(X \indep W | Y,Z)$ hold
\item But $(X \indep Y,W | Z)$ fails
\end{itemize}
\end{solution}
\end{exercisebox}

\begin{exercisebox}{Exercise 6: Markov Inequality}
Prove: For non-negative RV $X$ and $t > 0$, $\prob{X \geq t} \leq \frac{\expect{X}}{t}$.

\begin{solution}
Define indicator $I_{X \geq t}$:
\begin{align*}
\expect{X} &\geq \expect{tI_{X \geq t}} \\
&= t\prob{X \geq t} \\
\Rightarrow \prob{X \geq t} &\leq \frac{\expect{X}}{t}
\end{align*}
\end{solution}
\end{exercisebox}

\begin{exercisebox}{Exercise 7: Chebyshev's Inequality}
Prove using Markov: $\prob{|X-\mu| \geq k} \leq \frac{\sigma^2}{k^2}$.

\begin{solution}
Apply Markov to $Y = (X-\mu)^2$:
\begin{align*}
\prob{Y \geq k^2} &\leq \frac{\expect{Y}}{k^2} \\
\prob{(X-\mu)^2 \geq k^2} &\leq \frac{\sigma^2}{k^2} \\
\prob{|X-\mu| \geq k} &\leq \frac{\sigma^2}{k^2}
\end{align*}
\end{solution}
\end{exercisebox}

\end{document}
